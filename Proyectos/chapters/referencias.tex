\chapter{Bibliografía}
\label{ch:Bibliografia}

En este capítulo mostraremos cómo crear nuevas referencias bibliográficas, entradas en el glosario de términos y acrónimos y palabras para el índice.

\section{Referencias bibliográficas}
\label{s:referencias-bibliograficas}

Hay muchas formas diferentes de gestionar las referencias bibliográficas, así que aquí hemos decidido elegir una de ellas por considerarla la más cómoda y simple, que es mediante el paquete \textit{biblatex}.

El fichero de referencias, \texttt{references.bib}, incluirá una entrada por cada una de las referencias que se citan durante la memoria. Luego, en el cuerpo del texto, se podrán hacer referencias a dichas entradas y será \LaTeX~después quien se encargue de indexar correctamente, crear los hipervínculos y maquetar automáticamente.

El fichero \texttt{references.bib} puede tener muchas más de las referencias que se citan en el cuerpo del texto. Sin embargo, sólo aparecerán las referencias que se citen en el texto.

\notebox{\textbf{No has dicho en ningún momento \textit{bibliografía}} Sí. Las referencias bibliográficas, también conocidas como lista de referencias o simplemente referencias, son todas aquellas fuentes bibliográficas que han sido citadas a lo largo del documento. La bibliografía, también conocida como referencias externas, es simplemente una lista de recursos utilizados, citados o no. Como generalmente los no referenciados no se usan para dar soporte a un texto científico se suelen descartar.}
 
\subsection{¿Cómo creamos nuevas referencias?}



\begin{lstlisting}[language=tex,caption=Estructura general de una referencia]
@tipo{id,
    author = "Autor",
    title = "Título de la referencia (libro, artículo, enlace, ...)",
    campo1 = "valor",
    campo2 = "valor",
    \ldots
}
\end{lstlisting}

En esta entrada, \texttt{@tipo} indica el tipo de elemento (p. ej. \texttt{@article} para artículos o \texttt{@book} para libros) e \texttt{id} es un identificador \textbf{único en todo el documento} para el elemento. El resto de campos dependerán del tipo de la referencia, aunque generalmente casi todos los tipos comparten los campos de \texttt{author}, \texttt{title} o \texttt{year}.

AQUÍ CONTAR CÓMO SE AÑADEN ENTRADAS, LAS POSIBLES OPCIONES Y EL ENLACE A DONDE SE DESCRIBE TODO EN PROFUNDIDAD

\subsection{¿De qué manera puedo citar las referencias?}

AQUÍ CONTAR LAS DIFERENTES OPCIONES PARA REFERENCIAR ARTÍCULOS. PONGO LA BÁSICA, QUE ES~\cite{mcculloch1943logical}, PARA QUE ME APAREZCA EL CAPÍTULO DE REFERENCIAS.

\section{Referencias cruzadas}

\section{Referencias a recursos externos}

\section{Glosario y acrónimos}
\label{s:glosario}
Para gestionar el glosario y los acrónimos se hace uso del paquete \texttt{glossaries}. Es, quizá, algo complejo de configurar ya que permite muchas opciones diferentes.

Aquí proponemos una configuración por defecto para que lo único que haya que hacer sea añadir y referenciar las entradas.

\subsection{Definiendo los términos del glosario}

Un listado aquí todo majo:

\begin{lstlisting}[language=TeX]
\newglossaryentry{ex}{
    name={sample},
    description={an example}
}
\end{lstlisting}

\section{Índice}