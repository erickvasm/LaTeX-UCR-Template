\chapter{Conclusión}
\label{ch: Conclusión}


Como bien se ha ejemplificado en los puntos anteriores, se ha podido explicar
satisfactoriamente cada punto que se propusieron a un inicio del documento. Es
importante poder recalcar en este apartado las ideas más importantes para poder
clarificar y demarcar estos conceptos. La realización de este proyecto nos permitió
adquirir conocimientos sobre cómo trabaja el análisis de componentes principales en un entorno matemático y estadístico con énfasis educativo.

En relación con lo antes expuesto, se puede observar como el PCA brinda de resultados más significativos con menor cantidad de componentes, brindando una  serie de oportunidades que le permitan desarrollarse en un mejor estudio, no dejando de lado que hay algunos componentes con los cuales se descartan por su poca importancia, en los resultados de los sujetos evaluados.

Una de las ventajas del PCA es que nos permite una tabla de datos de n dimensiones, poder reducirla en las n dimensiones. Es decir, se creará una matriz de componentes con mismo número de variables, en nuestro caso son 7 variables las cuales eran: \textbf{tarea 1, tarea 2, tarea 3, parcial 1, parcial 2, exposición y proyecto}. También para la elaboración de gráficos detallados del análisis de componentes principales se hizo uso de la herramienta Past la cual además de poder realizar el análisis como excel brinda gráficos muy detallados.

Como pudimos observar en la sección \ref{ch: Análisis de resultado}, se pudo reducir de 7 variables a 2 componentes los cuales, estos dos componentes cuentan con un total del 75.32\% del total de datos de la matriz original. Al utilizar PCA se reduce en tan solo dos columnas la mayoría de datos, pero estamos perdiendo un porcentaje de 24.68\% de datos el cual es el precio a pagar por utilizar este método.

Como se puede observar en las tablas y gráficas anteriores se denota que el componente 1 y componente 2 corresponden a un mayor porcentaje de los resultados significativos, en donde estos componentes representan la tarea 1 y tarea 2 correspondientemente, dando como conclusión que los sujetos evaluados van presentando una creciente deserción  de trabajos a evaluar durante el curso.    