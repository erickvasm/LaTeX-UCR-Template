\chapter{Análisis de resultados}
\label{ch: Análisis de resultado}

Como datos principales de las notas \ref{tab:notas} del curso MA0321, el proceso que se realizó para obtener la tabla de datos principales fue sustituir las p variables originales por una nueva variable, z1, que resuma óptimamente la información. Esto supone que la nueva variable debe tener globalmente máxima correlación con las originales

Así mismo, la condición para que podamos prever con la mínima pérdida de información los datos observados, es utilizar la variable de máxima variabilidad. La cual para encontrar una una segunda variable z2, incorrelada con la anterior, y que tenga varianza máxima.\cite{CastilloGonzalez}

A continuación se presenta el cuadro de correlación:

\begin{table}[H]
\centering
\begin{tabular}{|c|c|c|c|c|c|c|c|}
\hline
\textbf{}     & \textbf{comp. 1} & \textbf{comp. 2} & \textbf{comp. 3} & \textbf{comp. 4} & \textbf{comp. 5} & \textbf{comp 6} & \textbf{comp 7} \\ \hline
\textbf{T1}   & -0,7977    & -0,3669    & -0,1405    & -0,3310    & -0,2576    & -0,0255     & -0,1804   \\ \hline
\textbf{T2}   & -0,7074    & -0,6408    & -0,0568      & 0,1599     & -0,0460     & 0,1091     & 0,2144     \\ \hline
\textbf{T3}   & -0,8428    & 0,2907     & 0,1288      & -0,3356    & 0,0562     & -0,1906    & 0,1905     \\ \hline
\textbf{Exp}  & -0,7437    & 0,3926     & 0,4203     & 0,1114     & -0,2523    & 0,1990     & -0,0139    \\ \hline
\textbf{P1}   & -0,6569      & 0,4072     & -0,5586    & 0,2587     & -0,1398    & -0,0599    & 0,0168     \\ \hline
\textbf{P2}   & -0,8636    & 0,1057     & -0,1447     & -0,1095    & 0,3999     & 0,2108     & -0,0747    \\ \hline
\textbf{Proy} & -0,8372    & -0,1790    & 0,2665     & 0,3290     & 0,1491     & -0,2232    & -0,1248     \\ \hline
\end{tabular}
\caption{Cuadro de covarianza}
    \label{tab:Cuadro de covarianza}
\end{table}

De acuerdo a nuestra tabla de datos originales, vamos a convertirlos en componentes, que van a ser los mismos en ambas. La idea es que los individuos que están en un espacio de n-dimensiones, se puedan proyectar en un plano. \ref{fig:Plano principal} 

Por lo cual, con los \textbf{comp. 1} al \textbf{comp. 7}, logramos graficar la tabla o plano principal. Además, los datos que se tienen en la tabla son coordenadas, para poder graficar, como se menciona anteriormente. 

Como un dato para el plano principal, es que logramos ver cuales componentes se agrupan o tienen similitudes, o por el contrario, están solitarios. Para poder ilustrar dicha tabla, tenemos el siguiente plano:


Como se puede observar en el gráfico anterior de correlaciones, donde se evalúan los componentes 2 con respecto al componente 1, que representan un 75.32\% de los datos, los puntos que están más cerca del borde del círculo están mejor representados que los que están un poco más lejos, además de ser los componentes con mayor información, se pueden observar las relaciones entre componentes y el conglomerado entre puntos.  



