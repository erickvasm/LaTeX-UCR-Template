\chapter{Introducción}
\label{ch:introduccion}

En la actualidad los datos representan uno de los activos mas importantes, los cuales permiten un sin número de cosas, tales como analizar tendencias o predecirlas con margenes de error muy pequeños, con una tasa de crecimiento exponencial, es imposible analizarlos uno a uno, ya que para eso existen técnicas como el PCA.

Por lo tanto en el presente trabajo se aborda de manera minuciosa todo lo relacionado con el análisis de componentes principales, desde cuestionamientos a que es o de donde viene, hasta como aplicarlo, para así poder obtener información comprensible al extraerla de un conjunto grande de datos, siendo posible representarla en una cantidad razonable de variables, pero estas siendo significativas para representar el conjunto entero de datos, sin perder información valiosa.

También cabe destacar que el análisis de componentes principales puede tener varios enfoques, en la presente investigación abarcamos un poco de lo que seria el enfoque estadístico que busca reducir la cantidad de datos obtenidos a valores que puedan representar el conjunto, pudiendo así hacer uso de diversas herramientas para mostrar los resultados obtenidos, los cuales suelen tener representaciones gráficas ayudando aun mas a la comprensión.

Por ultimo luego de una ardua investigación se desarrolla una propuesta para el análisis de componentes principales, sobre el conjunto de notas finales del curso MA-0321(calculo diferencial e integral) impartido en el primer semestre del año 2020, donde se aplicarán los conocimientos adquiridos mediante la presente investigación para realizar el análisis y la compresión de los datos obtenidos mediante el PCA realizado.






\newpage

\section{Objetivo General}

Describir el uso  de la técnica de análisis de componentes principales para el estudio estadístico y matemático en áreas como la informática.

\subsection{Objetivos específicos}

\begin{enumerate}
    \item Evidenciar las características que cuentan el análisis de componentes principales y sus diferentes aplicaciones con las que cuenta.

    \item Describir la propuesta de estudio para realizar el correspondiente estudio con el fin de extraer características cuantitativas y cualitativas del mismo.

    \item Aplicar el estudio de  análisis de componentes principales por medio de datos de estudiantes de la carrera de Informática empresaria, al acceso y uso de las tecnologías de información y comunicación.
\end{enumerate}