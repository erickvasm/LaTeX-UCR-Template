\chapter{Justificación}

La presente investigación tiene como motivo enfatizar en una de las herramientas o estrategias utilizada mayormente de forma estadística y de análisis de datos, surgidas a consecuencia de los grandes avances desde que el matemático Karl Pearson se inquietabá por lograr hallar conjuntos de datos de n-dimensiones en los cuales encargarán en líneas y planos gráficos de forma que los datos que tuvieran un mayor volumen, pudieran ser tratados en los campos de la informática y tratamiento de datos. El gran paso que realizo Pearson haciendo mención a los ordenadores que no sería hasta 1944 su primera aparición por medio de la computadora analítica ENIAC. 

La imperante necesidad de la realización de esta investigación radica en que análisis en componentes puede ser utilizada como un vital soporte al momento de efectuar decisiones por medio una reducción de variables en los datos, lo cual significa un enorme beneficio para la comunidad científica al cual nos dirigimos que son aquellas personas que quieran o se encuentren involucradas en un contexto informático, matemático y científico, así también para las que deseen ampliar más sus conocimientos acerca de temas de la rama de ciencia de datos.

La importancia que se le atribuye a esta investigación en nuestro contexto consiste en poder erradicar el desconocimiento presente acerca del tema a tratar con la finalidad de que el público objetivo comprenda el concepto de este, así también como los aspectos que engloba, de forma que además de conocer su definición pueda identificar de una manera clara las grandes bondades que son posibles obtener si se implementa. 

Para concretar lo anteriormente mencionado, en esta investigación se proporcionará una base teórica al explicar y difundir los aspectos más relevantes en lo que se refiere al tema tratado de análisis en componentes principales.
