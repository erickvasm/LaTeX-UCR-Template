\chapter{Propuesta de trabajo}

En este capítulo se abarcará la propuesta del estudio de PCA a un conjunto de datos que creemos de interés. 

Después de haber comprendido los fundamentos teóricos y la aplicación con los que se cuenta el PCA \textit{(sección: \ref{ch:Marco Teórico})}. Se realizará un análisis previo de la población de estudio que hemos decidido escoger para realizar el respectivo estudio  a las diferentes variables. A continuación se presenta la propuesta para el estudio de análisis de componentes principales. 

\section{Planteamiento}

¿Cuáles son las características por medio el estudio de componentes principales que se pueden extraer de la recolección de las notas del curso Cálculo Diferencial e Integral (MA0321), que tienen los estudiantes de la Carrera de Informática Empresarial de la sede Guanacaste en la pandemia del COVID-19, del primer semestre del 2020?

Sin duda, el propósito de conocer cuáles son las situaciones con la que los estudiantes de la carrera de Informática Empresarial de la sede de Guanacaste se enfrentaron, se crea con la finalidad de disponer de una información que puede ser utilizada por cualquier persona que desee hacer un estudio para la comparación de notas de curso MA0321 cursado en pandemia, con respecto al mismo curso (MA0321) pero cursado antes de la pandemia, es decir años posteriores a 2020. 

\section{Alcances}
Este tema para desarrollar cuenta con un alcance de una población finita, el cual se podrá aplicar exclusivamente a estudiantes matriculados en el curso de Cálculo Diferencial e Integral MA0321 de la carrera de informática empresarial de la Universidad de Costa Rica, sede de Guanacaste, en el verano del 2020. Se delimita a este tiempo y espacio para evitar que las conclusiones que se obtengan de esta investigación se acudan a generalizaciones con respecto a los resultados hacia otra población o bien que se generalice a la misma población de estudio que se está realizando, pero en diferente tiempo. Las conclusiones que se lleguen serán válidas solo para este tiempo, espacio y población mencionado anteriormente.

\section{Recolección de datos}
La técnica de recolección de datos utilizada en la investigación consiste en los datos proporcionados por el profesor a la generación de estudiantes de las notas del curso MA0321 del año 2020. Nosotros los autores de este proyecto pertenecimos a esta generación de estudiantes. Cabe destacar que los datos de las notas de los estudiantes del curso MA0321 son completamente anónimas y no se expondrán los carnets de los estudiantes por seguridad a los mismos. La información brinda será tratada en forma confidencial y solo con fines estadísticos.

\section{Tipo de muestreo}
Muestreo no aleatorio, intencional. Se elige este tipo de muestra porque es la más cómoda en estas circunstancias del que presenta el curso y facilitar el proceso de recolección de información pertinente al caso de estudio estadístico, por el hecho de que se puede escoger quienes son para la muestra bajo rigurosos criterios de selección. 

A forma de resumen se propuso los siguientes aspectos para el análisis de componentes principales. 

\begin{itemize}
    \item \textbf{Población:} estudiantes matriculados en el curso de Cálculo Diferencial e Integral MA0321 de la carrera de informática empresarial de la Universidad de Costa Rica.
    \item \textbf{Espacio y Tiempo:} estudiantes de la sede de Guanacaste, en el primer semestre del 2020.
    \item \textbf{Estudio:} aplicación del análisis de componentes principales a las notas obtenidas en las diferentes evaluaciones del curso MA0321.
\end{itemize}

Puede consultar los datos de la tabla de notas en la sección de anexos \ref{tab:notas}, se muestran de forma anónima por medio un ID para cada estudiante.