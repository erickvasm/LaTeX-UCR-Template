\section{Sección Dos}
\label{sec:Seccion Dos}

Las bibliografías se administran en el archivo “references.bib”. Para llamar una cita de la bibliografía se realiza así: \textbf{\cite{einstein}}

Esto es un algoritmo, se presenta en ingles pero de igual forma la hay en español. 

Ejemplo de algoritmo:

\begin{algorithm}

    \caption{Algoritmo de Euclides}
    \label{euclides}
    \begin{algorithmic}[1]
       \Procedure{Euclides}{$a,b$} \Comment{Encuentra el máximo común divisor entre $a$ y $b$}
        \State $r\gets a \bmod b$
        \While{$r\not=0$} \Comment{Si $r$ es cero ya tendríamos la respuesta}
            \State $a \gets b$
            \State $b \gets r$
            \State $r \gets a \bmod b$
        \EndWhile\label{euclidesfinwhile}
        \State \textbf{return} $b$\Comment{El M.C.D es $b$}
        
    \EndProcedure
    \end{algorithmic}
    
\end{algorithm}

Se puede obtener más detalles de como realizar un Pseudocódigo en el siguiente enlace: \textbf{{\footnotesize \url{https://tecdigital.tec.ac.cr/revistamatematica/Libros/LaTeX/MoraW_BorbonA_LibroLaTeX.pdf}}}